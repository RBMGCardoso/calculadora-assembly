\documentclass{report}
\usepackage[portuguese]{babel}
\usepackage[T1]{fontenc}
\usepackage[margin=1.3in]{geometry}
\usepackage{titlesec}
\usepackage{changepage}

%f Flowcharts
\usepackage{tikz}
\usetikzlibrary{shapes.geometric, arrows}
\tikzstyle{startstop} = [rectangle, rounded corners, minimum width=3cm, minimum height=1cm, text centered, draw=black, fill=red!30, text width=30mm]
\tikzstyle{io} = [trapezium, trapezium left angle=70, trapezium right angle=110, minimum width=3cm, minimum height=1cm, text centered, draw=black, fill=blue!30, text width=20mm]
\tikzstyle{process} = [rectangle, minimum width=3cm, minimum height=1cm, text centered, draw=black, fill=orange!30, text width=30mm]
\tikzstyle{decision} = [diamond, minimum width=3cm, minimum height=1cm, text centered, draw=black, fill=green!30, aspect=2, text width=30mm]
\tikzstyle{arrow} = [thick, ->, >=stealth]

\renewcommand{\thefootnote}{\Roman{footnote}}

% Keywords command
\providecommand{\keywords}[1]
{
	\small	
	\textbf{\textit{Palavras-chave---}} #1
}


\title{Calculadora em Assembly x86\\
	\hfill \break
	{\Large Escola Superior de Tecnologia de Tomar,}\\
	{\Large Licenciatura de Engenharia Informática}\\
	{\large Arquitetura de Computadores}}
\date{Janeiro, 2022}
\author{
	Rúben B. M. Gomes Cardoso\\
	Nº 23885
	\and 
	Rodrigo Miguel Barreto Serra\\
	Nº 24180
}

\begin{document}
	\maketitle
	
	\chapter*{Resumo}
No âmbito da unidade curricular Arquitetura de Computadores, foi solicitado um algoritmo em Assembly x86 (desenvolvido utilizando o Emu8086) que fosse capaz de calcular Divisões e Raízes Quadradas, utilizando valores fornecidos pelo utilizador através de uma interface gráfica que permitisse a escolha entre ambos os algoritmos previamente mencionados. O utilizador teria ainda a opção de optar por utilizar o teclado ou selecionar botões na interface de forma a escolher os valores pretendidos. \\

\keywords{Assembly x86, Divisão Inteira, Raiz Quadrada, Emu8086}
	\tableofcontents
	
	\chapter{Interface Gráfica}
\setlength{\parindent}{0pt}

\section{Objetivo}
Para permitir a obtenção dos dados escolhidos pelo utilizador foi necessário utilizar uma interface gráfica capaz de o fazer. Esta interface deveria permitir a escolha entre os dois algoritmos utilizando ou o rato ou o teclado como meio de input. A interface iria ainda precisar de mostrar os resultados dos algoritmos após a sua execução.

\section{Pseudocódigo}
\subsection{Tela inicial}
	\begin{enumerate}
		\item Imprimir um texto introdutório que indique o propósito do programa
		\item Desenhar os quadrados que representam botões onde o utilizador poderá clicar
		\item Ler o input do teclado que permite ao utilizador escolher se pretende utilizar o teclado ou o rato tanto para introduzir os inputs, como para escolher qual algoritmo pretende executar
	\end{enumerate}
\subsection{Tela de Input}
\begin{enumerate}
	\item Desenhar os 14 botões necessários para acomodar os inputs possíveis 
	\item Escrever o texto dentro dos botões todos os inputs possíveis\footnote{Algarismos de 0-9, \textbf{'-'}, \textbf{','}, backspace, confirmação}
	\item Mostrar o texto "Dividendo" ou "Radicando", de forma a indicar ao utilizador que pode inserir o valor pretendido
	\item Ler o teclado e adicionar os valores inseridos a um array
	\item Mostrar o array do passo acima consoante o utilizador adiciona ou remove valores
	\item Mostrar o texto "Divisor" (caso seja a tela de input da divisão), de forma a indicar ao utilizador que pode inserir o valor pretendido
	\item Ler o teclado e o rato e adicionar os valores inseridos a um array
	\item Mostrar o array do passo acima consoante o utilizador adiciona ou remove valores
	\item Executar o respetivo algoritmo e mostrar o resultado juntamente com o texto "Resultado"\\
\end{enumerate}
\setcounter{footnote}{0} 
\section{Detalhes de funcionamento}
\subsection{Inserção de dígitos}
Para a inserção de dígitos foi utilizado a interrupção 16H que permite a leitura do teclado e devolve o código ASCII do caractere pressionado. Sabendo o código da tecla pressionada basta verificar se este está dentro do intervalo aceite, neste caso, dígitos de 0-9 e caso se encontre dentro de cujo intervalo adiciona-se o algarismo\footnote{Visto que os códigos ASCII dos dígitos 0-9 começam no 48 é necessário subtrair esse valor ao código original de forma a adicionar o digito correto ao array} ao respetivo array.  \\

Por motivos de compatibilidade com o algoritmo da divisão previamente desenvolvido foi necessário armazenar o divisor em formato de número e não só dentro de um array. Para desenvolver esta funcionalidade, foi criada uma nova variável que armazenaria o divisor, onde iriamos concatenar \footnote{Utilizando $var = var\times 10 + digito$}os dígitos inseridos á medida que fossem selecionados.

\subsection{Visualização do input}
Para que o utilizador conseguisse visualizar os dados inseridos utilizou-se a interrupção 21H para imprimir os dígitos na janela da aplicação. Sempre que o utilizador inseria um digito no respetivo array, esse mesmo valor era escrito utilizando a interrupção 21H.

\subsection{Backspace}
No desenvolvimento da funcionalidade do Backspace foi utilizada a interrupção 21H novamente. Visto que impressão do caractere do backspace não remove o caractere anterior e ao invés disso, volta uma casa atrás, podemos utilizar esta funcionalidade para recuar um digito, imprimir um espaço\footnote{De forma a remover o caractere onde se encontra o cursor} e imprimir novamente um backspace para recuarmos uma casa.\\

Porém, remover o ultimo caractere impresso não remove o mesmo do array onde estão a ser armazenados todos os dígitos escolhidos, para resolver esse problema basta decrementar a posição atual do array por 1. Desta forma, o próximo digito inserido irá substituir no array o caractere que foi eliminado. \\

Visto que ainda temos a variável que armazena o divisor em formato de número, é necessário dividir esta variável por 10 de forma a remover o digito mais á direita, ou seja, o ultimo digito adicionado.

\titleformat{\subsection}[runin]
{\normalfont\large\bfseries}{\thesubsection}{1em}{}
\subsection{Negativo} \hfill \textcolor{red}{\underline{Esta funcionalidade é apenas utilizada no algoritmo da divisão}}\\ \\
Visto que numa divisão é possível a utilização de valores negativos foi necessário implementar um método que permitisse simbolizar que os números eram negativos. A norma é colocar o sinal de negativo antes do número e para facilitar a implementação foi decidido que se deveria limitar o input para que apenas fosse possível inserir o '\textbf{-}'. antes de qualquer digito.\\

Para fazer esta verificação, basta confirmar que o tamanho do array do dividendo/divisor\footnote{Dependendo de qual deles está a ser inserido} é igual a 0, querendo isto dizer que o array está vazio e que o '\textbf{-}' será o primeiro caractere.\\

É ainda necessário impedir a inserção de dois '\textbf{-}', para isto foi utilizada a flag implementada no algoritmo da divisão que indica se um número é negativo ou não. Considerando que o dividendo é o primeiro valor inserido, este só pode ter a flag a 1 ou 0, caso seja negativo ou não respetivamente. Logo se $flag = 1$ impedimos que o utilizador coloque outro '\textbf{-}', utilizando a mesma lógica para o divisor.


\titleformat{\subsection}[block]
{\normalfont\large\bfseries}{\thesubsection}{1em}{}

\section{Fluxograma}
\subsection{ShowMainScreen}
Procedimento responsável por desenhar e receber input da tela inicial do programa.\\
\fbox{
\begin{tikzpicture}[node distance=2cm]
	% Nodos
	\node (start) [startstop] {Inicio};
	\node (pr1) [process, right of=start, xshift=2cm] {WelcomeWindow};
	\node (pr2) [process, right of=pr1, xshift=2cm] {InsideSquareText};
	\node (pr3) [process, right of=pr2, xshift=2cm] {DesenhaQuadrados};
	\node (pr4) [process, below of=pr3, yshift=0.5cm] {RecebeMouse-KeyboardInput};
	\node (pr5) [process, left of=pr4, xshift=-2cm] {InputScreen};
	\node (end) [startstop, left of=pr5, xshift=-2cm] {Fim};
	
	% Setas
	\draw [arrow] (start) -- (pr1) ;
	\draw [arrow] (pr1) -- (pr2) ;
	\draw [arrow] (pr2) -- (pr3) ;
	\draw [arrow] (pr3) -- (pr4) ;
	\draw [arrow] (pr4) -- (pr5) ;
	\draw [arrow] (pr5) -- (end) ;
\end{tikzpicture}}

\subsection{WelcomeWindow}
Procedimento que imprime o texto de introdução ao programa.\\
\fbox{
	\begin{tikzpicture}[node distance=2cm]
		% Nodos
		\node (start) [startstop] {Inicio};
		\node (pr1) [process, right of=start, xshift=2cm] {Adiciona linhas de formar a alinhar welcomeString ao centro da tela};
		\node (pr2) [process, right of=pr1, xshift=2cm] {Imprime a welcome-String};
		\node (pr3) [process, right of=pr2, xshift=2cm] {Adiciona 2 linhas};
		\node (pr4) [process, below of=pr3, yshift=0.5cm] {Imprime a start-String};
		\node (end) [startstop, left of=pr4, xshift=-2cm] {Fim};
		
		% Setas
		\draw [arrow] (start) -- (pr1) ;
		\draw [arrow] (pr1) -- (pr2) ;
		\draw [arrow] (pr2) -- (pr3) ;
		\draw [arrow] (pr3) -- (pr4) ;
		\draw [arrow] (pr4) -- (end) ;
\end{tikzpicture}}

\subsection{InsideSquareText}
Procedimento que adiciona o texto de escolha dos algoritmos dentro dos botões desenhados posteriormente.\\
\fbox{
	\begin{tikzpicture}[node distance=2cm]
		% Nodos
		\node (start) [startstop] {Inicio};
		\node (pr1) [process, right of=start, xshift=2cm] {Adiciona linhas de formar a alinhar buttonString á posição desejada};
		\node (pr2) [process, right of=pr1, xshift=2cm] {Imprime a button-String};
		\node (end) [startstop, right of=pr2, xshift=2cm] {Fim};
		
		% Setas
		\draw [arrow] (start) -- (pr1) ;
		\draw [arrow] (pr1) -- (pr2) ;
		\draw [arrow] (pr2) -- (end) ;
\end{tikzpicture}}
\pagebreak
\subsection{DesenhaQuadrados}
Procedimento onde são desenhados os quadrados/botões que permitem uma visualização de onde clicar quando se utiliza o mouse como método de input.\\
\fbox{
	\begin{tikzpicture}[node distance=2cm]
		% Nodos
		\node (start) [startstop] {verticalLines};
		\node (pr1) [process, right of=start, xshift=2cm] {Inicializa as variáveis com os valores pretendidos};
		\node (pr2) [process, right of=pr1, xshift=2cm] {Desenha um pixel na posição (x,y)};
		\node (pr3) [process, right of=pr2, xshift=2cm] {Incrementa $y$};
		\node (dec1) [decision, below of=pr3,text width=20mm] {$y$<$larguraLim$};
		\node (end) [startstop, left of=dec1, xshift=-2cm] {call horizontalLines};
		
		\node (2start) [startstop, below of=start, yshift=-2cm] {horizontalLines};
		\node (2pr1) [process, right of=2start, xshift=2cm] {Inicializa as variáveis com os valores pretendidos};
		\node (2pr2) [process, right of=2pr1, xshift=2cm] {Desenha um pixel na posição (x,y)};
		\node (2pr3) [process, right of=2pr2, xshift=2cm] {Incrementa $x$};
		\node (2dec1) [decision, below of=2pr3,text width=17mm] {x< comprimentoLim};
		\node (2end) [startstop, left of=2dec1, xshift=-2cm] {Fim};
		
		% Setas
		\draw [arrow] (start) -- (pr1) ;
		\draw [arrow] (pr1) -- (pr2) ;
		\draw [arrow] (pr2) -- (pr3) ;
		\draw[arrow] (pr3) -- (dec1);
		\draw[arrow] (dec1) -- node[anchor=east]{sim}(pr2);
		\draw[arrow] (dec1) -- node[anchor=north]{não}(end);
		
		\draw [arrow] (2start) -- (2pr1) ;
		\draw [arrow] (2pr1) -- (2pr2) ;
		\draw [arrow] (2pr2) -- (2pr3) ;
		\draw[arrow] (2pr3) -- (2dec1);
		\draw[arrow] (2dec1) -- node[anchor=east]{sim}(2pr2);
		\draw[arrow] (2dec1) -- node[anchor=north]{não}(2end);
		
		\draw[arrow] (end) -| (2start);
\end{tikzpicture}} 

\subsection{RecebeMouseKeyboardInput}
Procedimento responsável por receber primeiramente, o método de input desejado e em seguida (dependendo do método escolhido), verificar qual tecla do teclado foi pressionada ou em qual dos botões desenhados o utilizador clicou.\\
\fbox{
\begin{tikzpicture}[node distance=2cm]
		% Nodos
		\node(start)[startstop]{Inicio};
		\node(pr1)[process, right of=start, xshift=2cm]{Iniciar inter-rupção do teclado};
		\node(dec1)[decision, right of=pr1, xshift=2cm, text width=10mm]{Escolha válida?};
		\node(dec2)[decision, right of=pr2, xshift=1.5cm, text width=13mm]{Valor escolhido};
		\node(pr2)[process, below of=dec2]{Iniciar inter-rupção do teclado};
		\node(pr3)[process, below of=pr2]{Iniciar inter-rupção do mouse};
		\node(pr4)[process, left of=pr2, xshift=-2cm]{Verificar input};
		\node(pr5)[process, left of=pr3, xshift=-2cm]{Verificar \\ botão do rato};
		\node(pr6)[process, left of=pr5, xshift=-2cm]{Verificar \\ posição do rato};
		\node(end)[startstop, left of=pr6, xshift= -1.5cm, yshift=2cm]{Final};
		
		% Setas
		\draw[arrow] (start) -- (pr1);
		\draw[arrow] (pr1) -- (dec1);
		\draw[arrow] (dec1) -- node[anchor=west]{não}(8, 1) -| (pr1);
		\draw[arrow] (dec1) -- node[anchor=south]{sim}(dec2);
		\draw[arrow] (dec2) -- node[anchor=east]{1}(pr2);
		\draw[arrow] (dec2) -- node[anchor=south]{2}(13.5, 0) |- (pr3);
		\draw[arrow] (pr2) -- (pr4);
		\draw[arrow] (pr3) -- (pr5);
		\draw[arrow](pr5) -- (pr6);
		\draw[arrow] (pr4) -- node[anchor=south, text width=30mm]{Input escolhido = tecla pressionada}(end);
		\draw[arrow] (pr6) -| node[anchor=north, text width=30mm]{Input escolhido = botão clicado}(end);
\end{tikzpicture}}
\pagebreak
\titleformat{\subsection}[runin]
{\normalfont\large\bfseries}{\thesubsection}{1em}{}
\subsection{KbHandlerInputScreen} \hfill \textcolor{red}{\underline{Exemplo do input da divisão}}
\titleformat{\subsection}[block]
{\normalfont\large\bfseries}{\thesubsection}{1em}{}\\ \\
Procedimento que processa e verifica todo o input, dentro da InputScreen. Capaz de, voltar à tela principal utilizando o Escape, adicionar sinais de negativo às respetivas variáveis utilizando "-", adicionar dígitos aos arrays e imprimir esses mesmos dígitos utilizando as teclas 0-9, eliminar o ultimo digito inserido utilizando o backspace, e confirmar os dados inseridos utilizando o Enter.
\\ \fbox{
\begin{tikzpicture}[node distance=2cm]
	% Nodos
	\node(start)[startstop]{Inicio};
	\node(pr1)[process, right of=start, xshift=2cm]{Iniciar inter-rupção do teclado};
	\node(dec1)[decision, right of=pr1, text width=15mm, yshift=-2cm]{Input \\ selecionado};
	\node(pr2)[process, below of=dec1, xshift=-6cm]{Inicia modo gráfico};
	\node(pr3)[process, below of=pr2, yshift=0.5cm]{Reseta variáveis};
	\node(pr4)[process, below of=pr3, yshift=0.5cm]{Call \\ showMainScreen};
	\node(pr17)[process, below of=pr4, yshift=0.5cm]{Volta ao inicio};
	
	\node(pr5)[process, below of=dec1, xshift=-2cm]{Verifica se o input é para Divis ou Divid};
	\node(pr6)[process, below of=pr5, yshift=0.5cm]{Verifica se já possui um negativo};
	\node(pr7)[process, below of=pr6, yshift=0.5cm]{Adiciona caso não exista};
	\node(pr16)[process, below of=pr7, yshift=0.5cm]{Volta ao inicio};
	
	\node(pr8)[process, below of=dec1, xshift=2cm]{Verifica se o input é para Divis ou Divid};
	\node(pr9)[process, below of=pr8, yshift=0.5cm]{Adiciona input ao respetivo array};
	\node(pr10)[process, below of=pr9, yshift=0.5cm]{Imprime o input};
	\node(pr18)[process, below of=pr10, yshift=0.5cm]{Volta ao inicio};
	
	\node(pr11)[process, below of=dec1, xshift=6cm]{Verifica se o input é para Divis ou Divid};
	\node(pr12)[process, below of=pr11, yshift=0.5cm]{Imprime um backspace};
	\node(pr13)[process, below of=pr12, yshift=0.5cm]{Imprime um espaço};
	\node(pr14)[process, below of=pr13, yshift=0.5cm]{Imprime um backspace};
	\node(pr19)[process, below of=pr14, yshift=0.5cm]{Volta ao inicio};
	
	\node(dec2)[decision, below of=dec1, yshift=-6cm, text width=15mm]{Ultimo Input?};
	\node(pr15)[process, below of=dec2, xshift=2cm]{Imprime\\ próximo input};
	\node(end)[startstop, below of=dec2, xshift=-2cm]{Executa\\ respetivo algoritmo};
	
	% Setas
	\draw[arrow] (start) -- (pr1);
	\draw[arrow] (pr1) -| (dec1);
	\draw[arrow] (dec1) -- node[anchor=east]{Escape}(pr2);
	\draw[arrow] (dec1) -- node[anchor=east]{Menos}(pr5);
	\draw[arrow] (dec1) -- node[anchor=west]{0-9}(pr8);
	\draw[arrow] (dec1) -- node[anchor=west]{Backspace}(pr11);
	\draw[arrow] (dec1) -- node[anchor=south]{Enter}(6,-3.8) -- (dec2);
	\draw[arrow] (pr2) -- (pr3);
	\draw[arrow] (pr3) -- (pr4);
	\draw[arrow] (pr4) -- (pr17);
	\draw[arrow] (pr5) -- (pr6);
	\draw[arrow] (pr6) -- (pr7);
	\draw[arrow] (pr7) -- (pr16);
	\draw[arrow] (pr8) -- (pr9);
	\draw[arrow] (pr9) -- (pr10);
	\draw[arrow] (pr10) -- (pr18);
	\draw[arrow] (pr11) -- (pr12);
	\draw[arrow] (pr12) -- (pr13);
	\draw[arrow] (pr13) -- (pr14);
	\draw[arrow] (pr14) -- (pr19);
	\draw[arrow] (pr15) -- (pr19);
	
	\draw[arrow] (dec2) -- node[anchor=west]{Não}(pr15);
	\draw[arrow] (dec2) -- node[anchor=east]{Sim}(end);
	
\end{tikzpicture}}
	\chapter{Algoritmo da Divisão}

\section{Objetivo}
Este algoritmo pretende receber as variáveis, dividendo, e divisor (podendo estas ser negativas) e em seguida executar o algoritmo da divisão de forma a devolver o resultado da divisão, bem como, o resto da mesma.

\section{Pseudocódigo}
\begin{enumerate}
	\item Inicialização das variáveis necessárias\footnote{Este pseudocódigo assume que as variáveis Divisor e Dividendo já foram obtidas através da interface gráfica}
	\item Retirar o primeiro HighOrder do dividendo e atribuir o seu valor  á variável Resto
	\item Iterar as vezes necessárias até a operação $i\footnote{A variável $i$ é utilizada como variável de iteração em ciclos}\times Divisor > Resto$
	\item Após a condição ser satisfeita:
	\begin{enumerate}
		\item Utilizar o valor atual da variável $i$ caso, $i\times Divisor = Resto$
		\item Realizar o cálculo $i = i - 1$,caso, $i\times Divisor > Resto$
	\end{enumerate}
	\item Concatenar o valor de $i$ á variável Quociente
	\item Verificar se existem mais algarismos no dividendo
	\begin{enumerate}
		\item Caso existam, voltar ao passo 3. com o novo valor retirado do dividendo
		\item Caso não existam, obter o valor do resultado a partir da expressão, \\$Resto = Resto - (i\times Divisor)$
	\end{enumerate}
\end{enumerate}

\section{Fluxograma}


	\chapter{Algoritmo da Raiz Quadrada}

\section{Objetivo}
Este algoritmo tem como objetivo receber o valor do Radicando e em seguida calcular a
raiz quadrada do valor inserido, e devolver o seu resultado.

\section{Pseudocódigo}
\begin{enumerate}
	\item Inicializar as variáveis necessárias\footnote{Este pseudocódigo assume que a variável Radicando já foi obtida através da interface gráfica}
	\item Utilizar divisões consecutivas por 10 de forma a obter o número de dígitos do Radicando e armazenar o resultado em nAlgarismos
	\item Obter o número de pares possíveis através da formula, $nGrupos = \frac{nAlgarismos}{2} $
	\begin{enumerate}
		\item Caso o resto da divisão do passo anterior seja igual a 1, realizar\\ $nGrupos = nGrupos + 1$
	\end{enumerate}
	\item Verificar o valor de nGrupos
	\begin{enumerate}
		\item Caso existam múltiplos grupos, ou seja $nGrupos > 1$, obter o grupo de HighOrder através da fórmula, $HighOrder = \frac{Radicando}{10^{(nGrupos)}}$
		\item Caso exista um único grupo atribuir o valor do Radicando ao HighOrder
	\end{enumerate}
	\item Caso seja a primeira iteração, realizar $aux \times aux\footnote{Variável auxiliar inicializada a 0}$ de forma a encontrar o maior número possível que não ultrapasse o HighOrder e armazenar a variável valor da variável aux na variável Elevado
	\item Caso não seja a primeira iteração, concatenar o novo HighOrder á variável aux
	\begin{enumerate}
		\item Caso seja a primeira iteração saltar para o passo 9.
	\end{enumerate}
	\item Calcular $(2\times Elevado:a)\times a$, incrementado o valor de $a$ enquanto o resultado não for superior a aux
	\item Concatenar o valor de $a$ obtido no passo anterior á variável Elevado
	\item Parar a execução do algoritmo caso $nGrupos = 1$ e armazenar o valor de Elevado na variável Resultado
	\item Calcular $Aux = HighOrder - Elevado^{2}$ 
	\item Calcular $Radicando = Radicando - (HighOrder\times 10^{nGrupos})$, de forma a remover o grupo mais á esquerda.
	\item Saltar para o passo 4.
\end{enumerate}
	\chapter{Problemas no desenvolvimento}
\begin{itemize}
	\normalsize
	\setlength\parindent{24pt}
	
	\item Visto que os algoritmos da Divisão e Raiz previamente desenvolvidos não suportavam números decimais, foi escolhido não adicionar na tela de input o botão da '\textbf{,}'.
	\begin{adjustwidth*}{0.5cm}{0cm}
			\textcolor{blue}{Isto poderia ser corrigido adicionando uma variável que contasse o número de dígitos antes do ponto decimal e alterando os cálculos para suportar esta funcionalidade.\\}
	\end{adjustwidth*}
	
	\item À custa da variável de Divisor ser armazenada não num array mas como um número, não é possível utilizar valores superiores a 65536, visto que qualquer valor superior ao supramencionado irá dar overflow no programa. 
	\begin{adjustwidth*}{0.5cm}{0cm}
		\textcolor{blue}{A forma de resolver esta falha seria alterar o código de forma a armazenar a variável Divisor num array.\\}
	\end{adjustwidth*}

	\item Existem certos casos, nomeadamente no algoritmo da Divisão, onde a concatenação com 0 falha, visto que em números inteiros, os $0s$ à esquerda são ignorados.
	\begin{adjustwidth*}{0.5cm}{0cm}
		\textcolor{blue}{Uma possível forma de resolver este problema é armazenar as variáveis, onde pode ser necessário concatenar com um zero á esquerda, dentro de um array.\\}
	\end{adjustwidth*}

	\item Devido a um desentendimento sobre os objetivos do trabalho e uma compreensão tardia sobre o que realmente era solicitado, a tela de input funciona à base do teclado e não do cursor do mouse como terá sido pedido. 
	\begin{adjustwidth*}{0.5cm}{0cm}
		\textcolor{blue}{Uma solução seria suportar a interrupção 33H, responsável pelo rato, em simultâneo com a interrupção 16H do teclado.\\}
	\end{adjustwidth*}

	\item Por falta de melhor conhecimento na altura do desenvolvimento, a utilização dos dois métodos de input (Mouse e Teclado) é possível na tela inicial, porém  o utilizador tem primeiro que escolher o método desejado utilizando o teclado.
	\begin{adjustwidth*}{0.5cm}{0cm}
		\textcolor{blue}{Este problema seria evitado se fosse utilizada a interrupção do mouse enquanto não fosse detetado o clique de uma tecla\\}
	\end{adjustwidth*}
	
	\item Visto que os arrays onde são armazenados os dígitos necessários, são inicializados com um valor fixo de 10 casas, não é possível a utilização de números com mais de 10 algarismos.
	\begin{adjustwidth*}{0.5cm}{0cm}
		\textcolor{blue}{Uma possível solução, embora não ótima, seria inicializar o array com um número exagerado de casas de forma a permitir mais dígitos. Porém, esta solução não resolve o problema e seria mais correto simplesmente limitar o input a 10 casas. \\}
	\end{adjustwidth*}
\end{itemize}

\chapter*{Conclusão}
\normalsize
Apesar de existirem alguns problemas no programa final,  e da utilização de uma linguagem focada para uma arquitetura já bastante desatualizada, limitando assim as possibilidades dessa mesma linguagem, é possível afirmar que o programa se encontra num estado funcional utilizando uma grande porcentagem das funcionalidades solicitadas para o mesmo.  

	
	
\end{document}