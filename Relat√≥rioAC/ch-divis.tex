\chapter{Algoritmo da Divisão}

\section{Objetivo}
Este algoritmo pretende receber as variáveis, dividendo, e divisor (podendo estas ser negativas) e em seguida executar o algoritmo da divisão de forma a devolver o resultado da divisão, bem como, o resto da mesma.

\section{Pseudocódigo}
\begin{enumerate}
	\item Inicialização das variáveis necessárias\footnote{Este pseudocódigo assume que as variáveis Divisor e Dividendo já foram obtidas através da interface gráfica}
	\item Retirar o primeiro HighOrder do dividendo e atribuir o seu valor  á variável Resto
	\item Iterar as vezes necessárias até a operação $i\footnote{A variável $i$ é utilizada como variável de iteração em ciclos}\times Divisor > Resto$
	\item Após a condição ser satisfeita:
	\begin{enumerate}
		\item Utilizar o valor atual da variável $i$ caso, $i\times Divisor = Resto$
		\item Realizar o cálculo $i = i - 1$,caso, $i\times Divisor > Resto$
	\end{enumerate}
	\item Concatenar o valor de $i$ á variável Quociente
	\item Verificar se existem mais algarismos no dividendo
	\begin{enumerate}
		\item Caso existam, voltar ao passo 3. com o novo valor retirado do dividendo
		\item Caso não existam, obter o valor do resultado a partir da expressão, \\$Resto = Resto - (i\times Divisor)$
	\end{enumerate}
\end{enumerate}

\section{Fluxograma}

