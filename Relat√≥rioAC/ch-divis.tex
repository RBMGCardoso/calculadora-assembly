\chapter{Algoritmo da Divisão}

\section{Objetivo}
Este algoritmo pretende receber as variáveis, dividendo, e divisor (podendo estas ser negativas) e em seguida executar o algoritmo da divisão de forma a devolver o resultado da divisão, bem como, o resto da mesma.

\section{Pseudocódigo}
\begin{enumerate}
	\item Inicialização das variáveis necessárias\footnote{Este pseudocódigo assume que as variáveis Divisor e Dividendo já foram obtidas através da interface gráfica}
	\item Retirar o primeiro HighOrder do dividendo e atribuir o seu valor  á variável Resto
	\item Iterar as vezes necessárias até a operação $i\footnote{A variável $i$ é utilizada como variável de iteração em ciclos}\times Divisor > Resto$
	\item Após a condição ser satisfeita:
	\begin{enumerate}
		\item Utilizar o valor atual da variável $i$ caso, $i\times Divisor = Resto$
		\item Realizar o cálculo $i = i - 1$,caso, $i\times Divisor > Resto$
	\end{enumerate}
	\item Concatenar o valor de $i$ á variável Quociente
	\item Verificar se existem mais algarismos no dividendo
	\begin{enumerate}
		\item Caso existam, voltar ao passo 3. com o novo valor retirado do dividendo
		\item Caso não existam, obter o valor do resultado a partir da expressão, \\$Resto = Resto - (i\times Divisor)$
	\end{enumerate}
\end{enumerate}
\newpage
\footnotesize
\begin{adjustwidth}{-1.7cm}{-2cm}
	\section{Fluxograma}
	\fbox{
	\begin{tikzpicture}[node distance=2cm]	
		% Blocos
		\node (start) [startstop] {Inicio};
		\node (pr1) [process, below of=start] {Inicializar variáveis};
		\node (in1) [io, below of=pr1] {Receber Dividendo e Divisor};
		\node (pr2) [process, below of=in1] {Retirar HighOrder do dividendo};
		\node (pr3) [process, below of=pr2] {$Resto = HighOrder$};
		\node (pr4) [process, below of=pr3] {$i = 0$};
		\node (pr5) [process, below of=pr4] {$i$++};
		\node (pr6) [process, below of=pr5] {$aux = i\times Divisor$};
		\node (dec1) [decision, below of=pr6] {$i\times Divisor \geq Resto$};
		\node (pr7) [process, below of=dec1] {Utilizar $i$ atual};
		\node (pr8) [process, right of=dec1, xshift=2.5cm] {Utilizar $i$++};
		\node (pr9) [process, right of=pr7, xshift=2.5cm] {Concatenar $i$ ao quociente};
		\node (dec2) [decision, right of=pr9, xshift=2.5cm] {Algarismos do dividendo = 0};
		\node (pr10) [process, above of=dec2, yshift=0.5cm] {Utilizar o High-Order seguinte} ;
		\node (pr11) [process, right of=dec2, xshift=2.5cm] {Calcular resto final} ;
		\node (pr12) [process, above of=pr11] {$Resto = Resto Anterior - (i\times Divisor)$} ;
		\node (end) [startstop, above of = pr12] {Fim};
		
		% Setas
		\draw [arrow] (start) -- (pr1) ;
		\draw [arrow] (pr1) -- (in1) ;
		\draw [arrow] (in1) -- (pr2) ;
		\draw [arrow] (pr2) -- (pr3) ;
		\draw [arrow] (pr3) -- (pr4) ;
		\draw [arrow] (pr4) -- (pr5) ;
		\draw [arrow] (pr5) -- (pr6) ;
		\draw [arrow] (pr6) -- (dec1) ;
		\draw [arrow] (dec1) -- node[anchor=west]{igual}(pr7) ;
		\draw [arrow] (dec1) -- node[anchor=north]{maior}(pr8) ;
		\draw [arrow] (dec1) -- (-2,- 16) -- node[anchor=east]{$menor$}(-2, -12) -- (pr5);
		\draw [arrow] (pr7) -- (pr9);
		\draw [arrow] (pr8) -- (pr9);
		\draw [arrow] (pr9) -- (dec2);
		\draw [arrow] (dec2) -- node[anchor=west]{Não} (pr10);
		\draw [arrow] (dec2) -- node[anchor=north]{Sim}(pr11);
		\draw [arrow] (pr10) |- (pr3);;
		\draw [arrow] (pr11) -- (pr12);
		\draw [arrow] (pr12) -- (end);
		
	\end{tikzpicture}}
\end{adjustwidth}	

