\chapter{Interface Gráfica}
\setlength{\parindent}{0pt}

\section{Objetivo}
Para permitir a obtenção dos dados escolhidos pelo utilizador foi necessário utilizar uma interface gráfica capaz de o fazer. Esta interface deveria permitir a escolha entre os dois algoritmos utilizando ou o rato ou o teclado como meio de input. A interface iria ainda precisar de mostrar os resultados dos algoritmos após a sua execução.

\section{Pseudocódigo}
\subsection{Tela inicial}
	\begin{enumerate}
		\item Imprimir um texto introdutório que indique o propósito do programa
		\item Desenhar os quadrados que representam botões onde o utilizador poderá clicar
		\item Ler o input do teclado que permite ao utilizador escolher se pretende utilizar o teclado ou o rato tanto para introduzir os inputs, como para escolher qual algoritmo pretende executar
	\end{enumerate}
\subsection{Tela de Input}
\begin{enumerate}
	\item Desenhar os 14 botões necessários para acomodar os inputs possíveis 
	\item Escrever o texto dentro dos botões todos os inputs possíveis\footnote{Algarismos de 0-9, \textbf{'-'}, \textbf{','}, backspace, confirmação}
	\item Mostrar o texto "Dividendo" ou "Radicando", de forma a indicar ao utilizador que pode inserir o valor pretendido
	\item Ler o teclado e adicionar os valores inseridos a um array
	\item Mostrar o array do passo acima consoante o utilizador adiciona ou remove valores
	\item Mostrar o texto "Divisor" (caso seja a tela de input da divisão), de forma a indicar ao utilizador que pode inserir o valor pretendido
	\item Ler o teclado e o rato e adicionar os valores inseridos a um array
	\item Mostrar o array do passo acima consoante o utilizador adiciona ou remove valores
	\item Executar o respetivo algoritmo e mostrar o resultado juntamente com o texto "Resultado"\\
\end{enumerate}
\setcounter{footnote}{0} 
\section{Detalhes de funcionamento}
\subsection{Inserção de dígitos}
Para a inserção de dígitos foi utilizado a interrupção 16H que permite a leitura do teclado e devolve o código ASCII do caractere pressionado. Sabendo o código da tecla pressionada basta verificar se este está dentro do intervalo aceite, neste caso, dígitos de 0-9 e caso se encontre dentro de cujo intervalo adiciona-se o algarismo\footnote{Visto que os códigos ASCII dos dígitos 0-9 começam no 48 é necessário subtrair esse valor ao código original de forma a adicionar o digito correto ao array} ao respetivo array.  \\

Por motivos de compatibilidade com o algoritmo da divisão previamente desenvolvido foi necessário armazenar o divisor em formato de número e não só dentro de um array. Para desenvolver esta funcionalidade, foi criada uma nova variável que armazenaria o divisor, onde iriamos concatenar \footnote{Utilizando $var = var\times 10 + digito$}os dígitos inseridos á medida que fossem selecionados.

\subsection{Visualização do input}
Para que o utilizador conseguisse visualizar os dados inseridos utilizou-se a interrupção 21H para imprimir os dígitos na janela da aplicação. Sempre que o utilizador inseria um digito no respetivo array, esse mesmo valor era escrito utilizando a interrupção 21H.

\subsection{Backspace}
No desenvolvimento da funcionalidade do Backspace foi utilizada a interrupção 21H novamente. Visto que impressão do caractere do backspace não remove o caractere anterior e ao invés disso, volta uma casa atrás, podemos utilizar esta funcionalidade para recuar um digito, imprimir um espaço\footnote{De forma a remover o caractere onde se encontra o cursor} e imprimir novamente um backspace para recuarmos uma casa.\\

Porém, remover o ultimo caractere impresso não remove o mesmo do array onde estão a ser armazenados todos os dígitos escolhidos, para resolver esse problema basta decrementar a posição atual do array por 1. Desta forma, o próximo digito inserido irá substituir no array o caractere que foi eliminado. \\

Visto que ainda temos a variável que armazena o divisor em formato de número, é necessário dividir esta variável por 10 de forma a remover o digito mais á direita, ou seja, o ultimo digito adicionado.

\titleformat{\subsection}[runin]
{\normalfont\large\bfseries}{\thesubsection}{1em}{}
\subsection{Negativo} \hfill \textcolor{red}{\underline{Esta funcionalidade é apenas utilizada no algoritmo da divisão}}\\ \\
Visto que numa divisão é possível a utilização de valores negativos foi necessário implementar um método que permitisse simbolizar que os números eram negativos. A norma é colocar o sinal de negativo antes do número e para facilitar a implementação foi decidido que se deveria limitar o input para que apenas fosse possível inserir o '\textbf{-}'. antes de qualquer digito.\\

Para fazer esta verificação, basta confirmar que o tamanho do array do dividendo/divisor\footnote{Dependendo de qual deles está a ser inserido} é igual a 0, querendo isto dizer que o array está vazio e que o '\textbf{-}' será o primeiro caractere.\\

É ainda necessário impedir a inserção de dois '\textbf{-}', para isto foi utilizada a flag implementada no algoritmo da divisão que indica se um número é negativo ou não. Considerando que o dividendo é o primeiro valor inserido, este só pode ter a flag a 1 ou 0, caso seja negativo ou não respetivamente. Logo se $flag = 1$ impedimos que o utilizador coloque outro '\textbf{-}', utilizando a mesma lógica para o divisor.


\titleformat{\subsection}[block]
{\normalfont\large\bfseries}{\thesubsection}{1em}{}

\section{Fluxograma}
\subsection{ShowMainScreen}
\fbox{
\begin{tikzpicture}[node distance=2cm]
	% Nodos
	\node (start) [startstop] {Inicio};
	\node (pr1) [process, right of=start, xshift=2cm] {WelcomeWindow};
	\node (pr2) [process, right of=pr1, xshift=2cm] {InsideSquareText};
	\node (pr3) [process, right of=pr2, xshift=2cm] {DesenhaQuadrados};
	\node (pr4) [process, below of=pr3] {RecebeMouse-KeyboardInput};
	\node (pr5) [process, left of=pr4, xshift=-2cm] {InputScreen};
	\node (end) [startstop, left of=pr5, xshift=-2cm] {Fim};
	
	% Setas
	\draw [arrow] (start) -- (pr1) ;
	\draw [arrow] (pr1) -- (pr2) ;
	\draw [arrow] (pr2) -- (pr3) ;
	\draw [arrow] (pr3) -- (pr4) ;
	\draw [arrow] (pr4) -- (pr5) ;
	\draw [arrow] (pr5) -- (end) ;
\end{tikzpicture}}

\subsection{WelcomeWindow}
\fbox{
	\begin{tikzpicture}[node distance=2cm]
		% Nodos
		\node (start) [startstop] {Inicio};
		\node (pr1) [process, right of=start, xshift=2cm] {Adiciona linhas de formar a alinhar welcomeString ao centro da tela};
		\node (pr2) [process, right of=pr1, xshift=2cm] {Imprime a welcome-String};
		\node (pr3) [process, right of=pr2, xshift=2cm] {Adiciona 2 linhas};
		\node (pr4) [process, below of=pr3] {Imprime a start-String};
		\node (end) [startstop, left of=pr4, xshift=-2cm] {Fim};
		
		% Setas
		\draw [arrow] (start) -- (pr1) ;
		\draw [arrow] (pr1) -- (pr2) ;
		\draw [arrow] (pr2) -- (pr3) ;
		\draw [arrow] (pr3) -- (pr4) ;
		\draw [arrow] (pr4) -- (end) ;
\end{tikzpicture}}

\subsection{InsideSquareText}
\fbox{
	\begin{tikzpicture}[node distance=2cm]
		% Nodos
		\node (start) [startstop] {Inicio};
		\node (pr1) [process, right of=start, xshift=2cm] {Adiciona linhas de formar a alinhar buttonString á posição desejada};
		\node (pr2) [process, right of=pr1, xshift=2cm] {Imprime a button-String};
		\node (end) [startstop, right of=pr2, xshift=2cm] {Fim};
		
		% Setas
		\draw [arrow] (start) -- (pr1) ;
		\draw [arrow] (pr1) -- (pr2) ;
		\draw [arrow] (pr2) -- (end) ;
\end{tikzpicture}}

\subsection{InsideSquareText}
\fbox{
	\begin{tikzpicture}[node distance=2cm]
		% Nodos
		\node (start) [startstop] {Inicio};
		\node (pr1) [process, right of=start, xshift=2cm] {Adiciona linhas de formar a alinhar buttonString á posição desejada};
		\node (pr2) [process, right of=pr1, xshift=2cm] {Imprime a button-String};
		\node (end) [startstop, right of=pr2, xshift=2cm] {Fim};
		
		% Setas
		\draw [arrow] (start) -- (pr1) ;
		\draw [arrow] (pr1) -- (pr2) ;
		\draw [arrow] (pr2) -- (end) ;
\end{tikzpicture}}


\subsection{InsideSquareText}
\fbox{
	\begin{tikzpicture}[node distance=2cm]
		% Nodos
		\node (start) [startstop] {Inicio};
		\node (pr1) [process, right of=start, xshift=2cm] {Adiciona linhas de formar a alinhar buttonString á posição desejada};
		\node (pr2) [process, right of=pr1, xshift=2cm] {Imprime a button-String};
		\node (end) [startstop, right of=pr2, xshift=2cm] {Fim};
		
		% Setas
		\draw [arrow] (start) -- (pr1) ;
		\draw [arrow] (pr1) -- (pr2) ;
		\draw [arrow] (pr2) -- (end) ;
\end{tikzpicture}}
