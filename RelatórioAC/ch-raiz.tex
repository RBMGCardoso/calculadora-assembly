\chapter{Algoritmo da Raiz Quadrada}

\section{Objetivo}
Este algoritmo tem como objetivo receber o valor do Radicando e em seguida calcular a
raiz quadrada do valor inserido, e devolver o seu resultado.

\section{Pseudocódigo}
\begin{enumerate}
	\item Inicializar as variáveis necessárias\footnote{Este pseudocódigo assume que a variável Radicando já foi obtida através da interface gráfica}
	\item Utilizar divisões consecutivas por 10 de forma a obter o número de dígitos do Radicando e armazenar o resultado em nAlgarismos
	\item Obter o número de pares possíveis através da formula, $nGrupos = \frac{nAlgarismos}{2} $
	\begin{enumerate}
		\item Caso o resto da divisão do passo anterior seja igual a 1, realizar\\ $nGrupos = nGrupos + 1$
	\end{enumerate}
	\item Verificar o valor de nGrupos
	\begin{enumerate}
		\item Caso existam múltiplos grupos, ou seja $nGrupos > 1$, obter o grupo de HighOrder através da fórmula, $HighOrder = \frac{Radicando}{10^{(nGrupos)}}$
		\item Caso exista um único grupo atribuir o valor do Radicando ao HighOrder
	\end{enumerate}
	\item Caso seja a primeira iteração, realizar $aux \times aux\footnote{Variável auxiliar inicializada a 0}$ de forma a encontrar o maior número possível que não ultrapasse o HighOrder e armazenar a variável valor da variável aux na variável Elevado
	\item Caso não seja a primeira iteração, concatenar o novo HighOrder á variável aux
	\begin{enumerate}
		\item Caso seja a primeira iteração saltar para o passo 9.
	\end{enumerate}
	\item Calcular $(2\times Elevado:a)\times a$, incrementado o valor de $a$ enquanto o resultado não for superior a aux
	\item Concatenar o valor de $a$ obtido no passo anterior á variável Elevado
	\item Parar a execução do algoritmo caso $nGrupos = 1$ e armazenar o valor de Elevado na variável Resultado
	\item Calcular $Aux = HighOrder - Elevado^{2}$ 
	\item Calcular $Radicando = Radicando - (HighOrder\times 10^{nGrupos})$, de forma a remover o grupo mais á esquerda.
	\item Saltar para o passo 4.
\end{enumerate}

\newpage
\newpage
\footnotesize
\begin{adjustwidth}{-2.2cm}{-2cm}
	\section{Fluxograma}
	\fbox{
	\begin{tikzpicture}[node distance=2cm]
		% Blocos
		\node (start) [startstop] {Inicio};
		\node (pr1) [process, right of=start, xshift=1.5cm] {Inicializar variáveis};
		\node (in1) [io, right of=pr1, xshift=1.8cm] {Receber radicando};
		\node (pr2) [process, right of=in1, xshift=1.8cm] {Calcular nº de algarismos do radicando};
		\node (pr3) [process, right of=pr2, xshift=1.8cm] {$nGrupos = \frac{nAlgarismos}{2}$};
		\node (dec1) [decision, below of=pr3, yshift=-0.7cm] {Resto do calculo anterior = 1};
		\node (pr4) [process, left of=dec1, xshift=-2.5cm] {$nGrupos = nGrupos$-1};
		\node (dec2) [decision, left of=pr4, xshift=-2.2cm] {nGrupos=1};
		\node (pr5) [process, left of=dec2, xshift=-2.2cm, yshift=0.7cm] {$HighOrder = Radicando$};
		\node (pr6) [process, left of=dec2, xshift=-2.2cm, yshift=-0.7cm] {$HighOrder = \frac{Radicando}{10^{(nGrupos)}}$};
		
		\node (dec3) [decision, below of=start, yshift=-4cm]{Primeira iteração?};
		\node (pr7) [process, below of=dec3]{Calcular $aux = aux^2$ até não ultrapassar o HighOrder};
		\node (pr8) [process, right of=dec3, xshift=2.4cm]{Concatenar HighOrder a Aux};
		\node (pr9) [process, right of=pr8, xshift=1.5cm]{Concatenar uma variável \\ iteradora(i) a Elevado};
		\node (pr10) [process, right of=pr9, xshift=1.5cm] {Calcular $(2\times Elevado:i)\times i$, incrementando $i$ até não ultrapassar aux};
		\node (pr11) [process, right of=pr10, xshift=1.5cm] {Concatenar $i$ á variável Elevado};
		\node (dec4) [decision, below of=pr7]{Primeira iteração?};
		\node (dec5) [decision, right of=dec4, xshift=5cm, yshift=-2cm] {$nGrupos = 1$};
		\node (pr12) [process, right of=dec5, xshift=2.3cm]{$Aux = Elevado^2 - HighOrder$};
		\node (pr13) [process, right of=pr12, xshift=1.7cm]{Remover o grupo HighOrder mais à esquerda};
		\node (pr14) [process, below of=dec5]{$Resultado = Elevado$};
		\node (end) [startstop, below of=pr14] {Fim};
		
		% Setas
		\draw [arrow] (start) -- (pr1);
		\draw [arrow] (pr1) -- (in1);
		\draw [arrow] (in1) -- (pr2);
		\draw [arrow] (pr2) -- (pr3);
		\draw [arrow] (pr3) -- (dec1);
		\draw [arrow] (dec1)  -- (14.9, -4) -- node[anchor=south]{Não}(6.2, -4) -- (dec2);
		\draw [arrow] (dec1) -- node[anchor=south]{Sim}(pr4);
		\draw [arrow] (pr4) -- (dec2);
		\draw [arrow] (dec2) -- node[anchor=south]{Sim}(pr5);
		\draw [arrow] (dec2) -- node[anchor=north]{Não}(pr6);
		\draw [arrow]  (pr5) -- (0, -2) -- (dec3);
		\draw [arrow]  (pr6) -- (0, -3.4) -- (dec3);
		\draw [arrow] (dec3) -- node[anchor=east]{Sim} (pr7);
		\draw [arrow] (pr7) -- (dec4);
		\draw [arrow] (dec3) --  node[anchor=south]{Não}(pr8);
		\draw [arrow] (pr8) -- (pr9);
		\draw [arrow] (pr9) -- (pr10);
		\draw [arrow] (pr10) -- (pr11);
		\draw [arrow] (dec4) |-  node[anchor=east]{Sim}(dec5);
		\draw [arrow] (dec4)  -|  node[anchor=north]{Não}(pr8);
		\draw [arrow] (pr11) |- (7,-7.5) -- (dec5);
		\draw [arrow] (dec5) --  node[anchor=south]{Não} (pr12);
		\draw [arrow] (pr12) -- (pr13);
		\draw [arrow] (pr13)  -| (17, -4.5) -| (6.2,-4) -- (dec2);
		\draw [arrow] (dec5) --  node[anchor=east]{Sim}(pr14);
		\draw [arrow] (pr14) -- (end);
		
		                                                                                
	\end{tikzpicture}}
\end{adjustwidth}