\chapter{Interface Gráfica}

\section{Objetivo}
Para permitir a obtenção dos dados escolhidos pelo utilizador foi necessário utilizar uma interface gráfica capaz de o fazer. Esta interface deveria permitir a escolha entre os dois algoritmos utilizando ou o rato ou o teclado como meio de input. A interface iria ainda precisar de mostrar os resultados dos algoritmos após a sua execução.

\section{Pseudocódigo}
\subsection{Tela inicial}
	\begin{enumerate}
		\item Imprimir um texto introdutório que indique o propósito do programa
		\item Desenhar os quadrados que representam botões onde o utilizador poderá clicar
		\item Ler o input do teclado que permite ao utilizador escolher se pretende utilizar o teclado ou o rato tanto para introduzir os inputs, como para escolher qual algoritmo pretende executar
	\end{enumerate}
\subsection{Tela da Divisão}
\begin{enumerate}
	\item Desenhar os 14 botões necessários para acomodar os inputs possíveis 
	\item Escrever o texto dentro dos botões todos os inputs possíveis\footnote{Algarismos de 0-9, \textbf{'-'}, \textbf{','}, backspace, confirmação}
	\item Mostrar o texto "Dividendo", de forma a indicar ao utilizador que pode inserir o valor pretendido
	\item Ler o teclado e o rato e adicionar os valores inseridos a um array
	\item Mostrar o array do passo acima consoante o utilizador adiciona ou remove valores
	\item Mostrar o texto "Divisor", de forma a indicar ao utilizador que pode inserir o valor pretendido
	\item Ler o teclado e o rato e adicionar os valores inseridos a um array
	\item Mostrar o array do passo acima consoante o utilizador adiciona ou remove valores
	\item Executar o algoritmo da divisão e mostrar o resultado juntamente com o texto "Resultado"
\end{enumerate}

\subsection{Tela da Raiz}
\begin{enumerate}
	\item 
\end{enumerate}